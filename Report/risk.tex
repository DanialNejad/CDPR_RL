\chapter{تحلیل سیستم و شناسایی اجزا}
\section{تحلیل سیستم حمل و نقل عمومی و شناسایی اجزا}

تحلیل سیستم حمل و نقل عمومی شامل شناسایی و بررسی دقیق اجزای مختلف این سیستم و چگونگی تعامل آن‌ها با یکدیگر است. این تحلیل به ما کمک می‌کند تا عملکرد کلی سیستم را بهبود دهیم و مشکلات را شناسایی و حل کنیم. در این بخش، به تفصیل اجزای کلیدی سیستم حمل و نقل عمومی و نقش هر یک از این اجزا را بررسی خواهیم کرد.

\section{وسیله‌های نقلیه عمومی}
وسایل نقلیه عمومی شامل اتوبوس‌ها، متروها، قطارها، ترامواها و تاکسی‌ها هستند که هر یک نقش مهمی در جابجایی مسافران دارند. ویژگی‌های کلیدی این وسایل نقلیه عبارتند از:

\textbf{ظرفیت حمل و نقل:}
 تعداد مسافرانی که وسیله نقلیه می‌تواند در هر سفر حمل کند.
 
\textbf{زمان‌بندی و فرکانس:}
تعداد دفعات حرکت وسیله نقلیه در طول روز و دقت زمانی در اجرای این برنامه.

\textbf{راحتی و امنیت:}
کیفیت خدمات ارائه شده به مسافران و میزان امنیت در طول سفر.

\section{ایستگاه‌ها و پایانه‌ها}
ایستگاه‌ها و پایانه‌ها نقاط کلیدی برای سوار و پیاده شدن مسافران هستند و شامل:

\textbf{ایستگاه‌های اتوبوس:}
محل‌هایی که اتوبوس‌ها در آن توقف می‌کنند تا مسافران سوار یا پیاده شوند.

\textbf{ایستگاه‌های مترو و قطار:}
شامل سکوها، ورودی‌ها و خروجی‌ها و امکانات رفاهی برای مسافران.

\textbf{پایانه‌های حمل و نقل:}
نقاط تجمیع وسایل نقلیه مختلف که امکان تعویض وسیله نقلیه را برای مسافران فراهم می‌کنند.

\section{زیرساخت‌های فیزیکی}
زیرساخت‌های فیزیکی شامل جاده‌ها، خطوط ریلی، تونل‌ها و پل‌ها هستند که وسایل نقلیه از آن‌ها استفاده می‌کنند. این زیرساخت‌ها باید:
- **با کیفیت بالا و قابل اطمینان**: برای کاهش خرابی‌ها و تأخیرها.
- **ایمن**: برای جلوگیری از حوادث و تصادفات.
- **مناسب برای همه اقشار جامعه**: به منظور دسترسی آسان برای افراد با نیازهای خاص.

\section{سیستم‌های اطلاعاتی و فناوری}
سیستم‌های اطلاعاتی و فناوری نقش مهمی در بهبود عملکرد و کارایی سیستم حمل و نقل عمومی دارند و شامل:

\textbf{سیستم‌های مدیریت ترافیک:}
برای کنترل و هدایت جریان ترافیک و کاهش ازدحام.

\textbf{سیستم‌های پرداخت الکترونیکی:}
برای ساده‌سازی فرایند خرید بلیت و افزایش راحتی مسافران.

\textbf{سیستم‌های اطلاع‌رسانی به مسافران:}
شامل نمایشگرهای دیجیتال و اپلیکیشن‌های موبایل برای ارائه اطلاعات لحظه‌ای در مورد زمان‌بندی و مسیرها.

\section{نیروی انسانی}
نیروی انسانی شامل رانندگان، نگهبانان، کارکنان تعمیر و نگهداری و پرسنل مدیریت است. ویژگی‌های کلیدی نیروی انسانی عبارتند از:

\textbf{مهارت و آموزش:}
کارکنان باید دارای مهارت‌های لازم و آموزش‌های کافی برای انجام وظایف خود باشند.

\textbf{رضایت شغلی و شرایط کاری مناسب:}
برای افزایش بهره‌وری و کاهش ترک خدمت.


