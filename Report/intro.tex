\chapter{مقدمه}
\pagenumbering{arabic}
سیستم حمل و نقل عمومی از اجزای حیاتی زیرساخت‌های شهری است که نقش کلیدی در بهبود کیفیت زندگی شهروندان ایفا می‌کند. این سیستم با ارائه خدمات جابجایی امن، کارآمد، و مقرون به صرفه به میلیون‌ها نفر در سراسر جهان کمک می‌کند تا به محل کار، تحصیل، و دیگر مقاصد خود برسند. حمل و نقل عمومی نه تنها به کاهش تراکم ترافیک و آلودگی هوا کمک می‌کند، بلکه به افزایش بهره‌وری اقتصادی و دسترسی بیشتر به فرصت‌های اجتماعی و اقتصادی نیز منجر می‌شود\cite{beirao2007}.


سیستم حمل و نقل عمومی از چندین جزء اصلی تشکیل شده است که هر یک نقشی اساسی در عملکرد کلی آن دارند. این اجزا شامل وسایل نقلیه (مانند اتوبوس‌ها، متروها، ترامواها و تاکسی‌ها)، زیرساخت‌ها (مانند ایستگاه‌ها، خطوط ریلی و مسیرهای اتوبوس)، و سیستم‌های پشتیبانی (مانند سیستم‌های بلیت‌فروشی، مدیریت ترافیک و مراکز تعمیر و نگهداری) می‌شود.


علی‌رغم مزایای فراوان، سیستم‌های حمل و نقل عمومی با چالش‌های متعددی نیز مواجه هستند. از جمله این چالش‌ها می‌توان به نیاز به تعمیر و نگهداری مستمر، تأمین مالی پایدار، مقابله با حوادث و بلایای طبیعی، و جلب رضایت مسافران اشاره کرد. با این حال، فرصت‌های بسیاری نیز برای بهبود و توسعه این سیستم‌ها وجود دارد. به‌کارگیری فناوری‌های پاک و پایدار، توسعه شبکه‌های حمل و نقل، و افزایش همکاری‌های بین‌المللی از جمله راهکارهایی هستند که می‌توانند به ارتقای سیستم‌های حمل و نقل عمومی کمک کنند.