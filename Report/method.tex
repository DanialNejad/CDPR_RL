\chapter{نتیجه‌گیری}
در این پروژه به بررسی و تحلیل ریسک‌های سیستم حمل و نقل عمومی از طریق روش‌های تحلیل درخت خطا (FTA) پرداخته شد. هدف اصلی این مطالعه شناسایی و ارزیابی عوامل ریسک زا و ریسک پذیر در سیستم حمل و نقل عمومی و ارائه راهکارهایی برای کاهش این ریسک‌ها بود. 
رویداد اصلی در این درخت خطا، "توقف کامل سیستم حمل و نقل" تعیین شد. عوامل مختلفی نظیر خرابی وسایل نقلیه، نگهداری نامناسب، کمبود نیروی کار ماهر، آموزش ناکافی کارکنان، برنامهریزی ناکارآمد مسیرها، و شرایط آب و هوایی نامناسب به عنوان رویدادهای ابتدایی شناسایی شدند. نمودار درخت خطا ترسیم شد که ارتباطات منطقی بین این رویدادها و رویداد اصلی را نمایش می‌داد. این تحلیل به شناسایی نقاط ضعف سیستم و ارائه راهکارهای پیشگیرانه کمک کرد.

\section{پیشنهادات برای بهبود}

با توجه به نتایج به دست آمده از تحلیلهای FTA، پیشنهادات زیر برای بهبود سیستم حمل و نقل عمومی ارائه میشود:

\subsection{بهبود نگهداری و تعمیرات:}
- تدوین و اجرای برنامههای نگهداری و تعمیرات پیشگیرانه برای کاهش احتمال خرابی وسایل نقلیه.
- استفاده از فناوریهای نوین برای پایش و مدیریت وضعیت وسایل نقلیه.

\subsection{آموزش و توسعه نیروی کار:}
- برگزاری دورههای آموزشی منظم و کاربردی برای کارکنان به منظور افزایش مهارتها و دانش فنی.
- استخدام و حفظ نیروی کار ماهر با ارائه انگیزهها و امکانات مناسب.

\subsection{برنامهریزی کارآمد:}
- بهبود فرآیندهای برنامهریزی مسیرها و زمانبندی سفرها برای افزایش کارایی و کاهش تأخیرات.
- استفاده از سیستمهای مدیریت هوشمند ترافیک برای بهینهسازی مسیرها و کاهش زمان سفر.

\subsection{مدیریت شرایط محیطی:}
- توسعه و اجرای برنامههای مدیریت بحران برای مواجهه با شرایط آب و هوایی نامناسب.
- ارتقاء زیرساختها برای افزایش مقاومت سیستم در برابر شرایط نامطلوب جوی.


این پروژه نشان داد که تحلیل ریسک با استفاده از روش FTA می‌تواند به شناسایی و ارزیابی نقاط ضعف و قوت سیستم حمل و نقل عمومی کمک کند. با بهره‌گیری از این تحلیل‌ها و اجرای راهکارهای پیشنهادی، می‌توان کارایی و پایداری سیستم حمل و نقل عمومی را بهبود بخشید و ریسک‌های مرتبط با آن را به حداقل رساند.